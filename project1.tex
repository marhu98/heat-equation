\documentclass[10pt,twoneside]{book}
%\documentclass[10pt,oneside]{book}
%\documentclass[14pt]{article}   	% use "amsart" instead of "article" for AMSLaTeX format
\pagestyle{headings}

\usepackage{natbib}
\usepackage{hyperref}

%\renewcommand\bibname{Bibliografía}
%\renewcommand\bibname{Bibliografy}

\usepackage{titlesec}


\usepackage{enumitem}

%\usepackage[spanish]{babel}

\titleformat{\chapter}[display]
  {\normalfont\bfseries}{}{0pt}{\Huge}

\usepackage{blindtext}
\usepackage[T1]{fontenc}
%\usepackage[utf8]{inputenc}

\usepackage{bbm}

\usepackage[margin=1in,footskip=.25in]{geometry}
%\usepackage[margin=1in]{geometry}
\usepackage{geometry}
\usepackage{mathtools}
\usepackage[latin9]{inputenc}
\usepackage[bottom]{footmisc}	% See geometry.pdf to learn the layout options. There are lots.
\geometry{letterpaper}
\usepackage{graphicx}
\usepackage{amssymb}
\usepackage{ragged2e}
\newcommand{\kfour}{\mathbb{K}^4}
\newcommand{\ktwo}{\mathbb{K}^2}
\newcommand{\pthree}{\mathbb{P}^3}
\newcommand{\afin}{\mathbb{A}}
\newcommand{\ov}{\overrightarrow}


\usepackage{tikz-cd}


\usepackage{faktor}


\usepackage{mathtools}


%\usepackage[spanish]{babel} % español
%\usepackage[utf8]{inputenc} % acentos sin codigo
%\usepackage{graphicx} % graficos
%\usepackage{amssymb}
%\usepackage{amsmath,amsthm}
%\usepackage{amsthm}

%\usepackage{hyperref}
%\usepackage{mathtools}
\usepackage{mathrsfs} % Letras caligráficas
\usepackage{bm}


\usepackage{tikz-cd}
\usepackage{xcolor}

\newcommand{\mcm}{\qopname \relax o{mcm}}

\newcommand{\massey}[3]{\langle[ {#1} ],[ {#2} ],[ {#3} ]\rangle}
\newcommand{\module}[1]{\vert #1 \lvert }
\newcommand{\dotprod}[2]{\langle #1,#2 \rangle}
\newcommand{\mbot}{{(\bot)}}
%\newcommand{\isoequal}{\stackrel{\text{iso}}{=}}
\newcommand{\isoequal}{\simeq}
\newcommand{\difequal}{\stackrel{\text{dif}}{=}}
\newcommand{\defequal}{\stackrel{\text{def}}{=}}
\newcommand{\isoapprox}{\stackrel{\text{iso}}{\displaystyle\approx}}
\newcommand{\Cech}{\v{C}ech}
\newcommand{\difpartial}[2]{\frac{\partial #1}{\partial #2}}
\newcommand{\difantipartial}[2]{\frac{\partial #1}{\overline{\partial} #2}}
\newcommand{\antipartial}{{\overline{\partial }}}

\newcommand{\comilla}{{}"{}}

\newcommand{\norm}[1]{\left\vert\left\vert #1 \right\vert\right\vert}

\newcommand{\e}[2]{e_{#1}^{(#2)}}

\newcommand{\f}[2]{f_{#1}^{(#2)}}


\usepackage{amsthm}

\newcounter{fakecnt}[section]
\def\thefakecnt{\arabic{section}}

\newtheorem{theorem}{Theorem}[chapter]
\newtheorem*{notation}{Notation}
\newtheorem{proposition}{Proposition}[chapter]
\newtheorem*{conjecture}{Conjecture}
\newtheorem{remark}{Remark}
\newtheorem{corollary}{Corollary}[chapter]
\newtheorem{lemma}{Lemma}[chapter]
\newtheorem{observation}{Observation}[chapter]
\newtheorem{example}{Example}[chapter]
%\renewcommand{proof}{\prof}
\theoremstyle{definition}
	\newtheorem{definition}{Definition}[chapter]


\newenvironment{hproof}{%
  \renewcommand{\proofname}{Esquema de demostraci�n}\proof}{\endproof}

 \newenvironment{prof}{%
  \renewcommand{\proofname}{Proof}\proof}{\endproof}


\newcommand{\ifff}{if and only if }


\renewcommand\qedsymbol{$QED$}

\newcommand{\overbar}[1]{\mkern 1.5mu\overline{\mkern-1.5mu#1\mkern-1.5mu}\mkern 1.5mu}
%\footskip = 0pt

\graphicspath{ {imgs/} }


\date{}

\usepackage{caption}

\begin{document}


First, if $\rho$ is an exact solution, then it's mass is preserved, meaning $m(t):=\int \rho(t,x)dx=\int \rho_0(x)dx = \text{cst}$.
Then conservation follows from the boundary conditions:
    $$
    m'(t)=\int_\mathbb{R} \partial_t\rho(t,x)dx,
    =\int_\mathbb{R} \partial_{xx}\rho(t,x)dx,
    =\partial_x\rho(t,\infty)-\partial_x\rho(t,-\infty),
    0
    $$

Now we center on the simpler problem:

$$
\begin{array}{l l}
    \partial_t \rho - \varepsilon^2 \Delta \rho &= 0 \text{ on } (0, 1) \times (0,1)\\
\rho(0,x)&=\rho_0(x),\;\forall x\in [0,1]\\
    \partial_x\rho(0,t)&=\partial_x\rho(1,t)=0,\;\forall t \in [0,1]\text{Neumann boundary conditions}
\end{array}
$$


To solve it we implement the Crank-Nicolson method, although we note that an explicit Euler, or implicit Euler in space would work as well.
Also, order one boundary conditions work just fine.

\begin{equation}
    \frac{\rho^{n+1}_i - \rho^n_i}{\Delta t} = \frac{\varepsilon^2}{2 (\Delta x)^2} \left[ \left( \rho^{n+1}_{i-1} -2\rho^{n+1}_i +\rho^{n+1}_{i+1} \right)
    + \left( \rho^{n}_{i-1} -2\rho^{n}_i +\rho^{n}_{i+1} \right) \right]
\end{equation}


Here $\rho(t_n,x_i)=\rho_i^n$.
Which in matrix form is:


\begin{equation}
\left(
\begin{array}{c c c c c c c}
-3 & 4& -1&0&\cdots&\cdots&0\\
-\alpha & 1+2\alpha & -\alpha & 0&\cdots&\cdots &0\\
0 &-\alpha & 1+2\alpha & -\alpha & \cdots &\cdots&0\\
\ddots&\ddots&\ddots&\ddots&\ddots&\ddots&\ddots\\
0 &\cdots&\cdots&0&-\alpha & 1+2\alpha & -\alpha \\
0 &\cdots&\cdots&0&1& -4& 3\\
\end{array}
\right)
\left(
\begin{array}{c}
\rho_{0}^{n+1}\\
\vdots\\
\rho_{i}^{n+1}\\
\rho_{i+1}^{n+1}\\
\vdots\\
\rho_{M}^{n+1}\\
\end{array}
\right)
=
\left(
\begin{array}{c c c c c c c}
0 & 0& 0&\cdots&\cdots&\cdots&0\\
\alpha & 1-2\alpha & \alpha & \cdots&\cdots&\cdots &0\\
0 &\alpha & 1-2\alpha & \alpha & \cdots &\cdots&0\\
\ddots&\ddots&\ddots&\ddots&\ddots&\ddots&\ddots\\
0 &\cdots&\cdots&\cdots&\alpha & 1-2\alpha & \alpha \\
0 &\cdots&\cdots&\cdots&\cdots& 0& 0\\
\end{array}
\right)
\left(
\begin{array}{c}
\rho_{0}^{n}\\
\vdots\\
\rho_{i}^{n}\\
\rho_{i+1}^{n}\\
\vdots\\
\rho_{M}^{n}\\
\end{array}
\right)
\end{equation}


To check that the mass is conserved by the method, we observe that the mass:
$
m(\rho)=\sum_{i=1}^{M-1} \rho_i
$
is preserved by the application of one iteration of the method.
Furthermore, by linearity, it enough to check on a basis (of the space of vectors which satisfies the boundary condition,
the method does not preserve the mass for any vector, this is clear from the observation $Le_1=0$):
Then we observe that $m(Ax)=m(x) \iff m(x)=m(A^{-1})$ in the case where $A$ is invertible.
So we have to check:
$$
\begin{aligned}
    m(L(e_2))=\cdots=m(L(e_{n-1})=1 \\
    m(R(e_2))=\cdots=m(R(e_{n-1})=1 \\
\end{aligned}
$$
{\bf Some other conditions}
Which are trivial to check.





\end{document}
